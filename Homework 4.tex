\documentclass{article}
%%%%%%%%%%%%%
% Loads packages
%%%%%%%%%%%%%
\usepackage[table]{xcolor}
\usepackage[utf8]{inputenc}
\usepackage[colorlinks=true,linkcolor=blue]{hyperref}
\usepackage{geometry} %package needed to set margins
\usepackage{fancyhdr}
\usepackage{graphicx}
\usepackage{amsmath}
\usepackage{amsthm}
\usepackage{mdframed}
\usepackage{tikz}
\usepackage{amsfonts}
\usepackage{listings}% http://ctan.org/pkg/listings
\lstset{
basicstyle=\ttfamily,
mathescape
}
\pagestyle{fancy}
\fancyhf{}
\chead{\textbf{Homework 3}}
\lhead{Math 213, Fall 2024}
\rhead{Due Sunday, 9/22 at 11:59pm}
%%%%%%%%%%%%%
% Sets margins
%%%%%%%%%%%%%
\newgeometry{left=1.5in,right=1in,top=1in,bottom=1in}
\setlength\headsep{3pt}
%%%%%%%%%%%%%
% Creates problem and solution environments
%%%%%%%%%%%%%
% Solution Environment
\newenvironment{solution}{\begin{proof}[Solution]}{\end{proof}}
% Problem Environment
\newenvironment{problem}[1]
{\begin{mdframed}[default]
\textbf{Problem #1:}
}
{\end{mdframed}
}
%%%%%%%%%%%
% Custom Commands
%%%%%%%%%%%
\newcommand{\gOne}{\cellcolor{green!50!white} 1}
\newcommand{\rZero}{\cellcolor{red!50!white} 0}
\begin{document}
\begin{problem}{\S 5.2: 8}
Suppose that a store offers gift certificates in denominations of $25$ and $40$
dollars. Determine the possible total amounts you can form using these gift
certificates. Prove your answer using strong induction.\newline


Solution:

The possible amounts of money we can obtain is $T=a\cdot25+b\cdot40$, where $a$ and $b$ are non-negative integer.
By observation, we found that both $25$ and $40$ are multiplier of $5$, so let's set the total amounts of money we can get as $5n$ where $n$ is the number of $5$ in such number.
for $n=5$, is $25$. for $n=8$, is $40$. for $n=10$, is $50$. for $n=13$, is $65$. for $n=15$, is $75$. for $n=16$, is $80$. for $n=18$, is $90$. for $n=20$, is $100$.
for $n=21$, is $105$. for $n=23$, is $115$. for $n=24$, is $120$. for $n=25$, is $125$. for $n=26$, is $130$. for $n=28$, is $140$. for $n=29$, is $145$. for $n=30$, is $150$. for $n=31$, is $155$. for $n=32$, is $160$.

Observe that for $n\geq28$, we can approach total amount of $T=5n$.

Thus, set $P(n)$ as for $n\geq28$, we can approach total amount of $T=5n$.

Basis step: $P(28)$: we can get it by $25+25+25+25+25+25+25+25+40$, so it is true.

Assume the inductive hypothesis, that $P(j)$ is true for all $j$ with $28\leq j\leq k$ where $k\geq 32$ then $k+1$ is true.
Since $k-4\geq 28$ so $P(k-4)$ is true, thus we can form $5k-20$ amount of money, then we add $25$ to it, we can get $5(k+1)$ amount of money. Thus, $P(k+1)$ is true.
The induction step satisfies. 

\end{problem}
\begin{problem}{\S 5.2: 10}
Assume that a chocolate bar consists of $n$ squares arranged in a rectangular
pattern. The entire bar, a smaller rectangular piece of the bar, can be broken
along on a vertical or horizontal line separating the squares. Assuming that only
one piece can be broken at a time, determine how many breaks you must successively
make to break the bar into $n$ separate squares. Use strong induction to prove your
answer.\newline

Solution:

Proposition: It takes $n-1$ breaks to separate it into $n$ squares. 

For the basis step, $n=1$ it does not need any break to have $1$ square appear. It is true.

For the inductive step, assume the proposition is true for $1\leq k\leq n$. Thus, for $k=n+1$, when we are breaking it, for the first time, we may get two situation.
The first situation is a bar with $a$ squares and a bar with $b$ squares; and the second situation is get a square and a bar with n squares.
For the first situation, by the inductive hypothesis, bar containing $a$ and $b$ can be break down by $a-1$ and $b-1$ time. And for the second situation, it can be break down by $n-1$ time.
And for both case, we calculate the total time of breaking down is $(a-1)+(b-1)+1=n$ and $(n-1)+1=n$.
Both are $n$ time, consists with the $P(n+1)$ that it should take $(n+1)-1=n$ step to breaking down.

\end{problem}
\begin{problem}{\S 6.1: 8}
How many different three-letter initials with none of the letters repeated can
people have?

Solution:

Since is a three letters initials and none of letters repeated. We first assume the letters are English characters, which is 26 in total.
Then, the total amount of the different three-letter initials will be $T=26\times25\times24=15600$.
\end{problem}
\begin{problem}{\S 6.1: 14}
How many bit strings of length $n$, where $n$ is a positive integer, start and end
with $1$s?

Solution:
Since the total length is $n$, and two of the bits are already $1$, the bit number we should deal with is $n-2$. And since it is bit strings, the type of digit on the bit is only $2$, which is $0$ and $1$.
Then, the amount of variation should be $2^{n-2}$.

\end{problem}
\begin{problem}{\S 6.1: 16}
How many strings are there of four lowercase letters that have the letter $x$ in
them?

Solution:
To determine how many strings are there of four lowercase letters that have the letter $x$ in them, we can first calculate the number of string which does not contain $x$ in it and the total number of string containing no matter weather $x$ in or not, and do the minus.

The total amount of string(no matter $x$ in or not):
\[T_{all}=26^4=456976\]

The amount of string does not containing $x$:
\[T_{no}=25^4=390625\]

The amount of string containing $x$:
\[T_{all}-T_{no}=456976-390625=66351\]

\end{problem}
\begin{problem}{\S 6.1: 26}
How many strings of four decimal digits
\begin{enumerate}
\item[(a)] do not contain the same digit twice?
\item[(b)] end with an even digit?
\item[(c)] have exactly three digits that are $9$s?

Solution:

(a) Since do not contain same digit twice, the total amount is $T=10\times9\times8\times7=5040$.

(b) We first determine the possible amount of number appearing on the last digit is $5$, which is $0$, $2$, $4$, $6$. $8$.
Then the total amount should be $T=10\times10\times10\times5=5000$.

(c) There are four possible position for the digit that is not $9$ be, and there will be $9$ possible number to be there, than $T=4\times9=36$

\end{enumerate}
\end{problem}
\begin{problem}{\S 6.1: 30}
How many license plates can be made using either three uppercase English letters
followed by three digits or four uppercase English letters followed by two digits?

Solution:

The total amount of possible combination for first type of license is $T_{1}=26^3\times10^3=17576\times1000$.
The total amount of possible combination for second type of license is $T_{middle}=26^4\times10^2=456976\times100$.
Total: $T=63273600$.
\end{problem}
\begin{problem}{\S 6.1: 36}
How many functions are there from the set $\{1, 2, \dots, n \}$, where $n$ is a
positive integer, to the set $\{ 0, 1 \}$?

Solution:

There are $2^n$, since every elements in the set $\{1,2,\dots ,n\}$ has two choice.

\end{problem}
\begin{problem}{\S 6.1: 37}
How many functions are there from the set $\{ 1, 2, \dots, n \}$, where $n$ is a
positive integer, to the set $\{ 0, 1 \}$
\begin{enumerate}
\item[(a)] that are one-to-one?
\item[(b)] that assign $0$ to both $1$ and $n$?
\item[(c)] that assign $1$ to exactly one of the positive integers less than
$n$?

Solution:

(a) There are no one-to-one function since the domain set is larger than the mapped set.

(b) The map of two element in the domain is already certify, then, the amount of possible functions for rest of the elements should be $T=2^{n-2}$.

(c) For this question, the amount of possible number that can be assign to $1$ is $n-1$, and for $n$ there are two possibility. So the total amount should be $2\cdot(n-1)$
\end{enumerate}
\end{problem}
\begin{problem}{\S 6.1: 40}
How many subsets of a set with $100$ elements have more than one element?

Solution:
For any set with n elements, the total number of possible subsets is 
$2^n$. This is because for each element in the set, there are two choices:

Include the element in a subset.
Exclude the element from a subset.
Since these choices are independent for each element, the total number of combinations (subsets) is calculated by multiplying the number of choices for each element.
Since there are $100$ set with one element, and a empty set which is the subset of all the set. The total number should be $T=2^{100}-101$
\end{problem}
\begin{problem}{\S 6.1: 44}
How many ways are there to seat four of a group of ten people around a circular
table where two seatings are considered the same when everyone has the same
immediate left and immediate right neighbor?

Solution:

The amount of possible combination for choosing $4$ people from $10$ is $T=10\times9\times8\times7=5040$.

And since when rotate, it considered as one combination, than divided by $4$
Then, the total amount of possible combination should be $T=5040/4=1260$.
\end{problem}
\begin{problem}{\S 6.2: 2}
Show that if there are 30 students in a class, then at least two have last names
that begin with the same letter.

Solution:

Since there are only 26 English characters, and the set with 26 items project to a set with 30 items must not be one to one, so it follows pigeonhole principle.
\end{problem}
\begin{problem}{\S 6.2: 6}
Let $d$ be a positive integer. Show that among any group of $d+1$ (not necessarily
consecutive) integers there are two with exactly the same remainder when they are
divided by $d$.

Solution:

Since there are only $d$ possible remainder when divided by $d$, then, by the pigeonhole principle, for a set with $d+1$ item, there must be at least 2 has the same remainder.
\end{problem}
\begin{problem}{\S 6.2: 8}
Show that if $f$ is a function from $S$ to $T$, where $S$ and $T$ are finite sets
with $|S| > |T|$, then there are elements $s_1$ and $s_2$ in $S$ such that $f(s_1)
= f(s_2)$, or in other words, $f$ is not one-to-one.

Solution:

Since the number of element of set $S$ is greater than number of element of set $T$, by pigeonhole principle, the function projecting $S$ to $T$ must not be one to one. $(k=|T|)$
\end{problem}
\begin{problem}{\S 6.2: 14}
\begin{enumerate}
\item[(a)] Show that if seven integers are selected from the first 10 positive
integers, there must be at least two pairs of these integers with the sum 11.
\item[(b)] Is the conclusion in (a) true if six integers are selected rather
than seven?

Solution:

(a) There are $5$ possible combination in first $10$ integer whose sum is $11$, which are $\{1,10\}$, $\{2,9\}$, $\{3,8\}$, $\{4,7\}$, $\{5,6\}$. To select $7$ number from the set of first $10$ positive integer. By the pigeonhole principle, at least 2 pairs will belong to the subset whose sum is $11$.

(b) No, since there only selected $6$ numbers, by the pigeonhole principle, only at least one will from the subset with sum $11$, but not $2$.

\end{enumerate}
\end{problem}
\end{document}