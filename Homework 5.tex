\documentclass{article}
%%%%%%%%%%%%%
% Loads packages
%%%%%%%%%%%%%
\usepackage[table]{xcolor}
\usepackage[utf8]{inputenc}
\usepackage[colorlinks=true,linkcolor=blue]{hyperref}
\usepackage{geometry} %package needed to set margins
\usepackage{fancyhdr}
\usepackage{graphicx}
\usepackage{amsmath}
\usepackage{amsthm}
\usepackage{mdframed}
\usepackage{tikz}
\usepackage{amsfonts}
\usepackage{listings}% http://ctan.org/pkg/listings
\lstset{
basicstyle=\ttfamily,
mathescape
}
\pagestyle{fancy}
\fancyhf{}
\chead{\textbf{Homework 5}}
\lhead{Math 213, Fall 2024}
\rhead{Due Sunday, 10/6 at 11:59pm}
%%%%%%%%%%%%%
% Sets margins
%%%%%%%%%%%%%
\newgeometry{left=1.5in,right=1in,top=1in,bottom=1in}
\setlength\headsep{3pt}
%%%%%%%%%%%%%
% Creates problem and solution environments
%%%%%%%%%%%%%
% Solution Environment
\newenvironment{solution}{\begin{proof}[Solution]}{\end{proof}}
% Problem Environment
\newenvironment{problem}[1]
{\begin{mdframed}[default]
\textbf{Problem #1:}
}
{\end{mdframed}
}
%%%%%%%%%%%
% Custom Commands
%%%%%%%%%%%
\newcommand{\gOne}{\cellcolor{green!50!white} 1}
\newcommand{\rZero}{\cellcolor{red!50!white} 0}
\begin{document}
\begin{problem}{\S 6.3 - 16}
How many subsets with an odd number of elements does a set with 10 elements have?

Solution:

\[T=C(10,1)+C(10,3)+C(10,5)+C(10,7)+C(10,9)\]
\[=\frac{10!}{1!\cdot(9!)}+\frac{10!}{3!\cdot(7!)}+\frac{10!}{5!\cdot(5!)}+\frac{10!}{7!\cdot(3!)}\frac{10!}{9!\cdot(1!)}\]
\[=10+120+252+120+10=512\]
\end{problem}
\begin{problem}{\S 6.3 - 20}
How many bit strings of length 10 have
\begin{enumerate}
\item[(a)] exactly three ``0''s?
\item[(b)] more ``0''s than ``1''s?
\item[(c)] at least seven ``1''s?
\item[(d)] at least three ``1''s?

Solution:

(a) \[T=C(10,3)=\frac{10!}{3!\cdot(10-3)!}=120\]
(b) \[T=C(10,10)+C(10,9)+C(10,8)+C(10,7)+C(10,6)\]
\[=\frac{10!}{10!\cdot(10-10)!}+\frac{10!}{9!\cdot(10-9)!}+\frac{10!}{8!\cdot(10-8)!}+\frac{10!}{7!\cdot(10-7)!}+\frac{10!}{6!\cdot(10-6)!}=386\]
(c)\[T=C(10,7)+C(10,8)+C(10,9)+C(10,10)\]
\[=\frac{10!}{10!\cdot(10-10)!}+\frac{10!}{9!\cdot(10-9)!}+\frac{10!}{8!\cdot(10-8)!}+\frac{10!}{7!\cdot(10-7)!}=176\]
(d)\[T=2^{10}-C(10,1)-C(10,2)-C(10,3)=968\]
\end{enumerate}
\end{problem}
\begin{problem}{\S 6.3 - 24}
How many ways are there for ten women and six men to stand in a line so that no two
men stand next to each other?

Solution:

Set 
\[T=P(11,6)\cdot P(10,10)={10!}\cdot \frac{11!}{(11-6)!}= 1207084032000\]
\end{problem}
\begin{problem}{\S 6.3 - 42}
Find a formula for the number of ways to seat $r$ of $n$ people around a circular
table, where seatings are considered the same if every person has the same two
neighbors without regard to which side these neighbors are sitting on.

Solution:

For $n=1$ or $n=2$, there is only one way to seat, for $n\geq 3$, the way is \[C(n,r)\cdot\frac{(r-1)!}{2} \]
\end{problem}
\begin{problem}{\S 6.4 - 10}
Give a formula for the coefficient of $x^k$ in the expansion of $(x+1/x)^{100}$,
where $k$ is an integer.

Solution:

Set the formula to be $(x+y)^{100}$, where $y=1/x$, than the expansion can be expressed by ${100 \choose j}x^{100-j}y^{j}$. Than substitute ${100 \choose j}x^{100-j}(1/x)^{j}={100 \choose j}x^{100-2j}$, since $x^{100-2j}=x^k$, $k=100-2j$, $j={(100-k)}/2$.
Now, we get \[{100 \choose \frac{100-k}{2}}x^k\], where\[{100 \choose \frac{100-k}{2}}\] is coefficient.\[=\frac{100!}{(\frac{100-k}{2})!\cdot(100-\frac{100-k}{2})!}\]

\end{problem}
\begin{problem}{\S 6.4 - 12}
The row of Pascal's triangle containing the binomia coefficients ${ 10 \choose k}$,
for $0 \leq k \leq 10$, is:
\[1~~10~~45~~120~~210~~252~~210~~120~~45~~10~~1 \]
Use Pascal's identity to produce the row immediately following this row in Pascal's
triangle.

Solution:

Add adjacent coefficient of the last row:
\[1~~11~~55~~165~~330~~462~~462~~330~~165~~55~~11~~1\]
\end{problem}
\begin{problem}{\S 6.4 - 22}
Prove the identity ${n \choose r}{r \choose k} = {n \choose k}{n-k \choose r-k}$,
whenever $n, r,$ and $k$ are non-negative integers with $r \leq n$ and $k \leq r$,
\begin{enumerate}
\item[(a)] using a combinatorial argument.
\item[(b)] using an argument based on the formula for the number of $r$-
combinations of a set with $n$ elements.

Solution:

(a) The left-hand side First, select an r-element subset from an n-element set $S$. This can be done in ${n \choose r}$ ways.
Then, from this r-element subset, select a k-element subset. This can be done in 
${r \choose k}$ ways.
Total ways: ${n \choose r}{r \choose k}$.
The right hand side first selected an k-element subset from n-element set represented as ${n \choose k}$, than choose $r-k$ elements in the remaining $n-k$ elements set which forms a set $r$
Both of choose a subset A with k elements and another, disjoint, subset with r - k elements.


(b)\[Left side = \frac{n!}{r!(n-r)!}\cdot\frac{r!}{k!(r-k)!}=\frac{n!}{k!(n-r)!(r-k)!}\]
\[Right side = \frac{n!}{k!(n-k)!}\cdot\frac{(n-k)!}{(r-k)!(n-r)!}=\frac{n!}{k!(n-r)!(r-k)!}\]

\end{enumerate}
\end{problem}
\begin{problem}{\S 6.4 - 27(a)}
Prove the \textbf{hockeystick identity}
\[ \sum_{k=0}^r {n+k \choose k} = {n + r + 1 \choose r } \]
whenever $n$ and $r$ are positive integers, using a combinatorial argument.

Solution:

Think about putting $r$ to $n+2$ boxes, there are \[{n+2+r-1\choose r}={n+r+1 \choose r}\] 
Which is the right side.

Then, we seperate $n+2$ boxes into $1$ and $n+1$ boxes. Then, we want to put $0\leq k \leq r$ into $n+1$ boxes which is\[{k+n+1-1 \choose k}={k+n \choose n}\]
There remaining $1$ box, which should put in the remaining $r-k$ balls in it. As the condition we choose a $k$ to put into the first $n+1$ boxes, there is only one way to put $r-k$ in to the remaining one.
So, the total way to make it is \[{k+n\choose k}\cdot 1={n+k \choose k}\]

Then, since the $k$ could change, that means, all the possible combination for putting $r$ into $n+2$ boxes equals to the sum of putting different $k$ from $0$ to $r$ into $n+1$ boxes, which is the sum:
\[\sum_{k=0}^r{n+k \choose k}={n+r+1 \choose r}\]
And it is the hockeystick identity.


\end{problem}
\begin{problem}{\S 6.5 - 10(a,c,d)}
A croissant shop has plain croissants, cherry croissants, chocolate croissants,
almond croissants, apple croissants, and broccoli croissants. How many ways are
there to choose
\begin{enumerate}
\item[(a)] a dozen croissants?
\item[(c)] two dozen croissants with at least two of each kind?
\item[(d)] two dozen croissants with no more than two broccoli croissants?
\end{enumerate}

Solution:

(a) \[{6+12-1 \choose 12}=6188\]

(c) Since two dozen, if we pick each at least $2$, we will get $12$ in total, which is one dozen, and other dozen we can choose randomly, which will have: \[{12+6-1 \choose 12}=6188\]

(d) For the case have at least three broccoli croissant, there are \[{6+21-1 \choose 21}\], for all case there are \[{6+24-1 \choose 24}\], The total number relevant to the question is: \[{6+24-1\choose 24}-{6+21-1\choose 21}=52975\]

\end{problem}
\begin{problem}{\S 6.5 - 20}
How many solutions are there to the inequality
\[ x_1 + x_2 + x_3 \leq 11, \]
where $x_1, x_2,$ and $x_3$ are non-negative integers?

Solution:

We can change it into $x_1+x_2+x_3+x_4=11$ and we put $11$ one into each box, than:
\[{11+4-1 \choose 11}=364\]

\end{problem}
\begin{problem}{\S 6.5 - 26}
How many positive integers less than $1,000,000$ have exactly one digit equal to
$9$ and have a sum of digits equal to 13?

Solution:

Since there are one digit equal to $9$, and sum of all digit should be $13$, there are $4$ ones we can put randomly, and since the number is less than 1000000, there should be $6$ digit for these numbers.
Except the one digit already assigned to $9$, there are $5$ digits. The combinations of those $5$ digits are \[{5+4-1 \choose 4}=70\] and since there are $6$ choice for positioning the $9$ in six digits, there are $6*70=420$ in total.

\end{problem}
\begin{problem}{\S 6.5 - 46}
A shelf holds 12 books in a row. How many ways are there to choose five books so
that no two adjacent books are chosen?

We first place $5$ books there and put $4$ in between, and there remaining $3$ books, than there are $6$ positions we can position it.
So, there are \[{6+3-1 \choose 3}=56\]

\end{problem}
\begin{problem}{\S 7.1 - 10}
What is the probability that a five-card poker hand contains the two of diamonds
and the three of spades?

Solution:

Since 2 of dimond and 3 of spades are fixed, we randomly choose rest 3 cards.
\[P=\frac{{50\choose 3}}{{52\choose 5}}=0.0075\]
\end{problem}
\begin{problem}{\S 7.1 - 12}
What is the probability that a five-card poker hand contains exactly one ace?

Solution:
Since there are 4 ace in total, and rest of 4 card we can choose randomly from 48 cards remaining.
\[\frac{4\cdot{48\choose 4}}{{52\choose 5}}\approx0.3\]
\end{problem}
\begin{problem}{\S 7.1 - 14}
What is the probability that a five-card poker hand contains cards of five
different kinds?

Solution:
Since there are 4 suits, there are $4^5$ in choosing suits.
Since there are 13 kinds, there are ${13\choose 5}$ for choosing valid kind.
So, in total:
\[\frac{4^5\cdot {13\choose 5}}{{52 \choose 5}}\approx 0.507\]
\end{problem}
\begin{problem}{\S 7.1 - 16}
What is the probability that a five-card poker hand contains a flush, that is, five
cards of the same suit?

Solution:
Since there are 4 suits, for same suits, there are $4$ in choosing suits.
Since there are 13 kinds, there are ${13\choose 5}$ possible combination for each suits.
So, in total:
\[\frac{4\cdot{13\choose 5} }{{52\choose 5}}\approx 0.00198\]
\end{problem}
\end{document}